\documentclass[12pt, letterpaper, onecolumn, conference, final]{IEEEtran}

\usepackage[margin = .5in]{geometry}
\usepackage{amsmath}
\usepackage{amsthm}
\usepackage{amssymb}
\usepackage{wasysym}
\usepackage{graphicx}
\usepackage{mathtools}

\title{Project Euler: Problem 44}
\author{Nathan Marianovsky}

\newcommand{\Mod}[1]{\ (\text{mod}\ #1)}

\theoremstyle{definition}
\newtheorem*{problem*}{Problem}
\newtheorem*{solution*}{Solution}

\theoremstyle{plain}

\renewcommand\thesection{\arabic{section}}

\begin{document}

\maketitle

\begin{center}
\fbox{
\begin{minipage}{7.3 in}
\begin{problem*}[Pentagon Numbers] 
Pentagonal numbers are generated by the formula, $P_n = \frac{n(3n-1)}{2}$. The first ten pentagonal numbers are:
\begin{equation*}
1, 5, 12, 22, 35, 51, 70, 92, 117, 145, \dots
\end{equation*}
It can be seen that $P_4 + P_7 = 22 + 70 = 92 = P_8$. However, their difference, $70 - 22 = 48$, is not pentagonal. Find the pair of pentagonal numbers, $P_j$ and $P_k$, for which their sum and difference are pentagonal and $D = |P_k - P_j|$ is minimised; what is the value of $D$?
\end{problem*}
\end{minipage}}
\end{center}

\vspace{.3cm}
\begin{solution*}
Given some arbitrary cap $N$, we want to find a Pentagonal pair such that both numbers are smaller than $N$. While the formula given is a great way for generating the Pentagon numbers, we also need a way to check as to whether or a not given number is a Pentagon number. This will be specifically useful in testing the addition and difference of two Pentagon numbers. To obtain $n$ use some algebraic manipulation:
\begin{equation*}
\begin{split}
P_n &= \frac{n(3n-1)}{2} \\
P_n &= \frac{3}{2}n^2 - \frac{1}{2}n \\
0 &= \frac{3}{2}n^2 - \frac{1}{2}n - P_n
\end{split}
\end{equation*}
Now using the quadratic formula:
\begin{equation*}
n = \frac{\frac{1}{2} \pm \sqrt{\frac{1}{4} + 6P_n}}{3} = \frac{1 \pm \sqrt{1 + 24P_n}}{6}
\end{equation*}
Since $n \in \mathbb{N}$ we can restrict ourselves to the positive square root. At the same time the numerator must satisfy:
\begin{equation*}
1 + \sqrt{1 + 24P_n} \Mod 6 \equiv 0
\end{equation*}
Now all the tools are ready. First generate the Pentagon numbers. Then take all pairs and find their sums and differences and check as to whether or not those are also Pentagon numbers. If both the sum and difference are Pentagon numbers, they are candidates for a solution. After finding all the candidates, find the one with the smallest difference which will correspond to the solution.
\end{solution*}



\end{document}