\documentclass[12pt, letterpaper, onecolumn, conference, final]{IEEEtran}

\usepackage[margin = .5in]{geometry}
\usepackage{amsmath}
\usepackage{amsthm}
\usepackage{amssymb}
\usepackage{wasysym}
\usepackage{graphicx}
\usepackage{mathtools}

\title{Project Euler: Problem 124}
\author{Nathan Marianovsky}

\newcommand{\Mod}[1]{\ (\text{mod}\ #1)}

\theoremstyle{definition}
\newtheorem*{problem*}{Problem}
\newtheorem*{solution*}{Solution}

\theoremstyle{plain}

\renewcommand\thesection{\arabic{section}}

\begin{document}

\maketitle

\begin{center}
\fbox{
\begin{minipage}{7.3 in}
\begin{problem*}[Ordered Radicals] 
The radical of $n$, rad($n$), is the product of the distinct prime factors of $n$. For example, 504 = $2^3 \times 3^2 \times 7$, so rad(504) = $2 \times 3 \times 7$ = 42. If we calculate rad($n$) for $1 \leq n \leq 10$, then sort them on rad($n$), and sorting on $n$ if the radical values are equal, we get:
\begin{center}
\def\arraystretch{1.5}
\begin{tabular}{| c | c |}
\hline
$n$ & rad($n$) \\ \hline
1 & 1 \\ \hline
2 & 2 \\ \hline
3 & 3 \\ \hline
4 & 2 \\ \hline
5 & 5 \\ \hline
6 & 6 \\ \hline
7 & 7 \\ \hline
8 & 2 \\ \hline
9 & 3 \\ \hline
10 & 10 \\ \hline
\end{tabular}
\hspace{.3cm} $\rightarrow$ \hspace{.3cm}
\begin{tabular}{| c | c | c |}
\hline
$n$ & rad($n$) & k \\ \hline
1 & 1 & 1 \\ \hline
2 & 2 & 2 \\ \hline
4 & 2 & 3 \\ \hline
8 & 2 & 4 \\ \hline
3 & 3 & 5 \\ \hline
9 & 3 & 6 \\ \hline
5 & 5 & 7 \\ \hline
6 & 6 & 8 \\ \hline
7 & 7 & 9 \\ \hline
10 & 10 & 10 \\ \hline
\end{tabular}
\end{center}
Let $E(k)$ be the $k$th element in the sorted $n$ column; for example, $E(4) = 8$ and $E(6) = 9$. If rad($n$) is sorted for $1 \leq n \leq 100000$, find $E(10000)$.
\end{problem*}
\end{minipage}}
\end{center}

\vspace{.3cm}
\begin{solution*}
It is best to approach this problem by first assuming that the list is of length $N$, though the problem specifically wants $N=100000$. For each value of $n$ in the table produced, we need the prime factorization \footnotemark:
\begin{equation*}
n = p_1^{m_1} p_2^{m_2} \dots p_l^{m_l}
\end{equation*}
so that:
\begin{equation*}
\text{rad}(n) = p_1 p_2 \dots p_l
\end{equation*}
After this it is a simple task of organizing the list based on rad($n$) first, then $n$. 
\end{solution*}

\footnotetext{See the personal comments for problem 3 for a proof of the Fundamental Theorem of Arithmetic}



\end{document}