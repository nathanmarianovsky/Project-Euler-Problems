\documentclass[12pt, letterpaper, onecolumn, conference, final]{IEEEtran}

\usepackage[margin = .5in]{geometry}
\usepackage{amsmath}
\usepackage{amsthm}
\usepackage{amssymb}
\usepackage{wasysym}
\usepackage{graphicx}
\usepackage{mathtools}

\title{Project Euler: Problem 137 Personal Comments}
\author{Nathan Marianovsky}

\theoremstyle{definition}
\newtheorem*{problem*}{Problem}
\newtheorem*{solution*}{Solution}

\theoremstyle{plain}

\renewcommand\thesection{\arabic{section}}

\begin{document}

\maketitle

\section*{\underline{\Large{Closed Form of Fibonacci Generating Function}}}
\vspace{.3cm}
\noindent
In the solution I provided, I made a big jump by saying that:
\begin{equation*}
A_F(x) = xF_1 + x^2F_2 + x^3F_3 + \dots = \frac{x}{1 - x - x^2}
\end{equation*}
which turns an infinite series into a finite closed form. But so where did this closed form come from? Before moving on lets define the generating function in a proper form:
\begin{equation*}
A_F(x) = \sum_{i=1}^\infty x^iF_i
\end{equation*}
Now observe what happens when we use the known recurrence relation for the Fibonacci sequence:
\begin{equation*}
\begin{split}
A_F(x) &= \sum_{i=1}^\infty x^iF_i \\
&= F_0 + xF_1 + \sum_{i=2}^\infty x^iF_i \\
&= F_0 + xF_1 + \sum_{i=2}^\infty x^i(F_{i-1} + F_{i-2}) \\
&= F_0 + xF_1 + \sum_{i=2}^\infty x^iF_{i-1} + \sum_{i=2}^\infty x^iF_{i-2} \\
&= F_0 + xF_1 + \sum_{k=1}^\infty x^{k+1}F_k + \sum_{k=0}^\infty x^{k+2}F_k \\
&= F_0 + x^2F_0 + xF_1 + x\sum_{k=1}^\infty x^kF_k + x^2\sum_{k=1}^\infty x^kF_k \\
&= x + xA_F(x) + x^2A_F(x)
\end{split}
\end{equation*}
where I exploited the Fibonacci numbers because placing the $F_0$ in the above calculations is meaningless since it is 0. Now use the above to solve for $A_F(x)$:
\begin{equation*}
\begin{split}
A_F(x) &= x + xA_F(x) + x^2A_F(x)  \\
A_F(x) (1 - x - x^2) &= x \\
A_F(x) &= \frac{x}{1 - x - x^2}
\end{split}
\end{equation*}




\end{document}