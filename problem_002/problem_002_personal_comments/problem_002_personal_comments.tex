\documentclass[12pt, letterpaper, onecolumn, conference, final]{IEEEtran}

\usepackage[margin = .5in]{geometry}
\usepackage{amsmath}
\usepackage{amsthm}
\usepackage{amssymb}
\usepackage{wasysym}
\usepackage{graphicx}
\usepackage{mathtools}

\title{Project Euler: Problem 2 Personal Comments}
\author{Nathan Marianovsky}

\newcommand{\Mod}[1]{\ (\text{mod}\ #1)}

\theoremstyle{definition}
\newtheorem*{problem*}{Problem}
\newtheorem*{solution*}{Solution}

\theoremstyle{plain}

\renewcommand\thesection{\arabic{section}}

\begin{document}

\maketitle

\section*{\underline{\Large{Fibonacci Sequence in Closed Form}}}
\vspace{.3cm}
\noindent
In the solution to this problem I only mentioned the closed form of the Fibonacci sequence and how to use it. Though this suffices for the problem, I believe it is important to know where results come from. So once more consider the Fibonacci sequence:
\begin{equation*}
0, 1, 1, 2, 3, 5, 8, 13, 21, 34, \dots
\end{equation*}
where this can be generated by the known recurrance relation:
\begin{equation*}
F_n = F_{n-1} + F_{n-2} \hspace{.3cm} \text{where} \hspace{.3cm} F_0 = 0 \hspace{.1cm} \text{and} \hspace{.1cm} F_1 = 1
\end{equation*}
Now to obtain the closed form we need to solve the above recurrence relation. Using similar techniques to the ones utilized in solving \textit{linear-constant coeffiecient-homogeneous} ordinary differential equations, we guess that:
\begin{equation*}
F_n = \alpha^n \hspace{.3cm} \text{where} \hspace{.3cm} \alpha \in \mathbb{R}
\end{equation*}
Plugging this into the recurrence relation and working out the algebra gives:
\begin{equation*}
\begin{split}
F_n &= F_{n-1} + F_{n-2} \\
\alpha^n &= \alpha^{n-1} + \alpha^{n-2} \\
0 &= \alpha^{n-2} \Big[ \alpha^2 - \alpha - 1 \Big] \\
\end{split}
\end{equation*}
If $\alpha^{n-2} = 0 \implies \alpha = 0$ which is the trivial case. The remaining quadratic equation is what is known as the characteristic equation for the recurrence relation. Solving for $\alpha$ using the quadratic formula gives:
\begin{equation*}
\alpha = \frac{1 \pm \sqrt{1 - 4(1)(-1)}}{2(1)} = \frac{1 \pm \sqrt{5}}{2}
\end{equation*}
Lets give each one a name:
\begin{equation*}
\phi = \frac{1 + \sqrt{5}}{2} \hspace{.3cm} \text{and} \hspace{.3cm} \psi = \frac{1 - \sqrt{5}}{2}
\end{equation*}
where $\phi$ is known as the \textit{golden ratio}. So now we have two linearly independent solutions. Using the superposition principle, the most general solution can be written as:
\begin{equation*}
F_n = c_1\phi^n + c_2\psi^n \hspace{.3cm} \text{where} \hspace{.3cm} c_1, c_2 \in \mathbb{R}
\end{equation*}
The only thing left to do is to figure out the coefficients using the known intial conditions:
\begin{equation*}
\begin{split}
F_0 = 0 &= c_1 + c_2 \\
F_1 = 1 &= c_1\phi + c_2\psi
\end{split}
\end{equation*}
Rewriting this system of equations as a matrix and solving gives:
\def\arraystretch{1.5}
\begin{equation*}
\begin{split}
\begin{pmatrix}
1 & 1 \\
\phi & \psi
\end{pmatrix} \begin{pmatrix}
c_1 \\ c_2
\end{pmatrix} &= \begin{pmatrix}
0 \\ 1
\end{pmatrix} \\
\begin{pmatrix}
c_1 \\ c_2
\end{pmatrix} &= \begin{pmatrix}
\frac{\psi}{\psi-\phi} & -\frac{1}{\psi-\phi} \\
-\frac{\phi}{\psi-\phi} & \frac{1}{\psi-\phi}
\end{pmatrix} \begin{pmatrix}
0 \\ 1
\end{pmatrix} = \begin{pmatrix}
-\frac{1}{\psi-\phi} \\ \frac{1}{\psi-\phi} 
\end{pmatrix} = \begin{pmatrix}
\frac{1}{\sqrt{5}} \\ -\frac{1}{\sqrt{5}}
\end{pmatrix}
\end{split}
\end{equation*}
Finally the closed form can be written as:
\begin{equation*}
F_n = \frac{1}{\sqrt{5}} (\phi^n - \psi^n)
\end{equation*}
Past this point I am going to just use some algebraic manipulation to arrive at the form I used in the solution:
\begin{equation*}
\psi = \frac{1 - \sqrt{5}}{2} \times \frac{1 + \sqrt{5}}{1 + \sqrt{5}} = -\frac{4}{2(1 + \sqrt{5})} = -\frac{1}{\frac{1 + \sqrt{5}}{2}} = -\frac{1}{\phi} = -\phi^{-1}
\end{equation*}
Plugging this into the closed form for $\psi$ finally gives the desired form:
\begin{equation*}
F_n = \frac{1}{\sqrt{5}} \Big( \phi^n - ( -\phi)^{-n} \Big)
\end{equation*}




\end{document}