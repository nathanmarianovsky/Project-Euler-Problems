\documentclass[12pt, letterpaper, onecolumn, conference, final]{IEEEtran}

\usepackage[margin = .5in]{geometry}
\usepackage{amsmath}
\usepackage{amsthm}
\usepackage{amssymb}
\usepackage{wasysym}
\usepackage{graphicx}
\usepackage{mathtools}

\title{Project Euler: Problem 12}
\author{Nathan Marianovsky}

\newcommand{\Mod}[1]{\ (\text{mod}\ #1)}

\theoremstyle{definition}
\newtheorem*{problem*}{Problem}
\newtheorem*{solution*}{Solution}

\theoremstyle{plain}

\renewcommand\thesection{\arabic{section}}

\begin{document}

\maketitle

\begin{center}
\fbox{
\begin{minipage}{7.3 in}
\begin{problem*}[Highly Divisible Triangular Number] 
The sequence of triangle numbers is generated by adding the natural numbers. So the $7^\text{th}$ triangle number would be 1 + 2 + 3 + 4 + 5 + 6 + 7 = 28. The first ten terms would be:
\begin{equation*}
1, 3, 6, 10, 15, 21, 28, 36, 45, 55, \dots
\end{equation*}
Let us list the factors of the first seven triangle numbers:
\begin{equation*}
\begin{split}
\textbf{1}:& 1 \\
\textbf{3}:& 1, 3 \\
\textbf{6}:& 1, 2, 3, 6 \\
\textbf{10}:& 1, 2, 5, 10 \\
\textbf{15}:& 1, 3, 5, 15 \\
\textbf{21}:& 1, 3, 7, 21 \\
\textbf{28}:& 1, 2, 4, 7, 14, 28
\end{split}
\end{equation*}
We can see that 28 is the first triangle number to have over five divisors. What is the value of the first triangle number to have over five hundred divisors?
\end{problem*}
\end{minipage}}
\end{center}

\vspace{.3cm}
\begin{solution*}
To approach this problem generalize first and assume that we are being asked for the first tirangular number to have at least $N$ divisors. Now generate the triangular numbers by adding sequential natural numbers:
\begin{equation*}
T = \Big\{ x \Big| x = \sum_{i=1}^n i \hspace{.3cm} \forall n \in \mathbb{N} \Big\}
\end{equation*}
and break each number down into its prime factorization:
\begin{equation*}
x = p_1^{m_1} p_2^{m_2} \dots p_k^{m_k}
\end{equation*}
At each $x$ check the number of distinct prime factors versus the required value $N$. If this number is at least $N$, we have found our solution, if not, continue the cycle until the requirement is met.
\end{solution*}



\end{document}