\documentclass[12pt, letterpaper, onecolumn, conference, final]{IEEEtran}

\usepackage[margin = .5in]{geometry}
\usepackage{amsmath}
\usepackage{amsthm}
\usepackage{amssymb}
\usepackage{wasysym}
\usepackage{graphicx}
\usepackage{mathtools}

\title{Project Euler: Problem 47}
\author{Nathan Marianovsky}

\newcommand{\Mod}[1]{\ (\text{mod}\ #1)}

\theoremstyle{definition}
\newtheorem*{problem*}{Problem}
\newtheorem*{solution*}{Solution}

\theoremstyle{plain}

\renewcommand\thesection{\arabic{section}}

\begin{document}

\maketitle

\begin{center}
\fbox{
\begin{minipage}{7.3 in}
\begin{problem*}[Distinct Primes Factors] 
The first two consecutive numbers to have two distinct prime factors are:
\begin{equation*}
\begin{split}
14 &= 2 \times 7 \\
15 &= 3 \times 5
\end{split}
\end{equation*}
The first three consectuive numbers to have three distinct prime factors are:
\begin{equation*}
\begin{split}
644 &= 2^2 \times 7 \times 23 \\
645 &= 3 \times 5 \times 43 \\
646 &= 2 \times 17 \times 19
\end{split}
\end{equation*}
Find the first four consecutive integers to have four distinct prime factors. What is the first of these numbers?
\end{problem*}
\end{minipage}}
\end{center}

\vspace{.3cm}
\begin{solution*}
The basic approach is to take consecutive integers and break them down into their prime factorization:
\begin{equation*}
a = p_1^{m_1} p_2^{m_2} \dots p_k^{m_k}
\end{equation*}
and count the number of distinct primes. Lets say to generalize that you want to find the first set of $k$ integers that have $k$ distinct prime factors, you check the current $k$ integers to see if they all meet the requirement, if not, move each one over by one and check again. Repeat the checking process until a result has popped up.
\end{solution*}



\end{document}