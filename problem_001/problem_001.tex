\documentclass[12pt, letterpaper, onecolumn, conference, final]{IEEEtran}

\usepackage[margin = .5in]{geometry}
\usepackage{amsmath}
\usepackage{amsthm}
\usepackage{amssymb}
\usepackage{wasysym}
\usepackage{graphicx}
\usepackage{mathtools}

\title{Project Euler: Problem 1}
\author{Nathan Marianovsky}

\DeclarePairedDelimiter\floor{\Bigg\lfloor}{\Bigg\rfloor}

\theoremstyle{definition}
\newtheorem*{problem*}{Problem}
\newtheorem*{solution*}{Solution}

\theoremstyle{plain}

\renewcommand\thesection{\arabic{section}}

\begin{document}

\maketitle

\begin{center}
\fbox{
\begin{minipage}{7.3 in}
\begin{problem*}[Multiples of 3 and 5] 
If we list all the natural numbers below 10 that are mutliples of 3 or 5, we get 3, 5, 6, and 9. The sum of these multiples is 23. Find the sum of all multiples of 3 or 5 below 1000.
\end{problem*}
\end{minipage}}
\end{center}

\vspace{.3cm}
\begin{solution*}
In this problem specifically we want to find the sum below 1000, but this can be generalized to some cap $N$. There are two ways that this problem can be approached. First a list can be made of all the natural numbers below $N$ that follow the requirements where the sum of all the elements corresponds to the solution. The second approach is to find the values of $k_1$ and $k_2$ such that:
\begin{equation*}
\begin{split}
3k_1 &= N \\
5k_2 &= N
\end{split}
\end{equation*}
where the $k$'s correspond to how high the multiples of 3 and 5 can go before they go above the cap $N$. Their integer solutions are modeled by:
\begin{equation*}
\begin{split}
k_1 &= \floor{\frac{N}{3}} \\
k_2 &= \floor{\frac{N}{5}}
\end{split}
\end{equation*}
So now instead of taking every natural number starting from 1 and onwards checking whether it fits the requirements, the numbers can be generated into the following list:
\begin{equation*}
v = \Big\{3, 6, 9, \dots, 3k_1, 5, 10, 15, \dots, 5k_2\Big\}
\end{equation*}
where the generated list should not contain any repetitions of numbers that are divisible by both 3 and 5. So now the solution corresponds to:
\begin{equation*}
V = \sum_{i} v_i
\end{equation*}
\end{solution*}



\end{document}