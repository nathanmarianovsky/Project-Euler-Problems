\documentclass[12pt, letterpaper, onecolumn, conference, final]{IEEEtran}

\usepackage[margin = .5in]{geometry}
\usepackage{amsmath}
\usepackage{amsthm}
\usepackage{amssymb}
\usepackage{wasysym}
\usepackage{graphicx}

\title{Project Euler: Problem 31}
\author{Nathan Marianovsky}

\theoremstyle{definition}
\newtheorem*{problem*}{Problem}
\newtheorem*{solution*}{Solution}

\theoremstyle{plain}

\renewcommand\thesection{\arabic{section}}

\begin{document}

\maketitle

\begin{center}
\fbox{
\begin{minipage}{7.3 in}
\begin{problem*}[Coin Sums] 
In England the currency is made of pound, \pounds, and pence, p, and there are eight coins in general circulation:
\begin{equation*}
1\text{p}, 2\text{p}, 5\text{p}, 10\text{p}, 20\text{p}, 50\text{p}, \pounds 1 \hspace{.1cm} (100\text{p}), \pounds 2 \hspace{.1cm} (200\text{p})
\end{equation*}
It is possible to make \pounds2 in the following way:
\begin{equation*}
(1 \times \pounds 1) + (1 \times 50\text{p}) + (2 \times 20\text{p}) + (1 \times 5\text{p}) + (1 \times 2\text{p}) + (3 \times 1\text{p})
\end{equation*}
How many different ways can \pounds2 be made using any number of coins?
\end{problem*}
\end{minipage}}
\end{center}

\vspace{.3cm}
\begin{solution*}
This problem is equivalent to finding the number of different solutions satisfying:
\begin{equation*}
x_1 + 2x_2 + 5x_3 + 10x_4 + 20x_5 + 50x_6 + 100x_7 + 200x_8 = r \hspace{.3cm} \text{where} \hspace{.3cm} x_i \geq 0 \hspace{.3cm} \forall i \in \{1,2,\dots,8\} 
\end{equation*}
In this problem specifically we want $r=200$, but the calculations can be generalized to any given sum $r$. Now in order to solve this problem, consider the method of generating functions. Each variable will have a corresponding generating function that shows its possibilities:
\begin{center}
\def\arraystretch{2}
\begin{tabular}{| c | c | c |}
\hline
$x_i$ & Generating Function & Simplified Form \\ \hline
$x_1$ & $g_{x_1}(t) = 1 + t + t^2 + \dots + t^r$ & $g_{x_1}(t) = \frac{1 - t^{r+1}}{1-t}$ \\ \hline
$x_2$ & $g_{x_2}(t) = 1 + t^2 + t^4 + \dots + t^r$ & $g_{x_2}(t) = \frac{1 - t^{r+2}}{1-t^2}$ \\ \hline
$x_3$ & $g_{x_3}(t) = 1 + t^5 + t^{10} + \dots + t^r$ & $g_{x_3}(t) = \frac{1 - t^{r+5}}{1-t^5}$ \\ \hline
$x_4$ & $g_{x_4}(t) = 1 + t^{10} + t^{20} + \dots + t^r$ & $g_{x_4}(t) = \frac{1 - t^{r+10}}{1-t^{10}}$ \\ \hline
$x_5$ & $g_{x_5}(t) = 1 + t^{20} + t^{40} + \dots + t^r$ & $g_{x_5}(t) = \frac{1 - t^{r+20}}{1-t^{20}}$ \\ \hline
$x_6$ & $g_{x_6}(t) = 1 + t^{50} + t^{100} + \dots + t^r$ & $g_{x_6}(t) = \frac{1 - t^{r+50}}{1-t^{50}}$ \\ \hline
$x_7$ & $g_{x_7}(t) = 1 + t^{100} + t^{200} + \dots + t^r$ & $g_{x_7}(t) = \frac{1 - t^{r+100}}{1-t^{100}}$ \\ \hline
$x_8$ & $g_{x_8}(t) = 1 + t^{200} + t^{400} + \dots + t^r$ & $g_{x_8}(t) = \frac{1 - t^{r+200}}{1-t^{200}}$ \\ \hline
\end{tabular}
\end{center}
Giving an overall generating function:
\begin{equation*}
\begin{split}
g(t) &= g_{x_1}(t)g_{x_2}(t)\dots g_{x_8}(t) \\
&= \frac{(1-t^{r+1})(1-t^{2r+2})(1-t^{5r+5})(1-t^{10r+10})(1-t^{20r+20})(1-t^{50r+50})(1-t^{100r+100})(1-t^{200r+200})}{(1-t)(1-t^2)(1-t^5)(1-t^{10})(1-t^{20})(1-t^{50})(1-t^{100})(1-t^{200})}
\end{split}
\end{equation*}
The solution to the problem now corresponds to finding the coefficient of $t^r$ in the power series expansion of the above generating function. For this problem specifically we want the coefficient of $t^{200}$ which happens to be 73,682 when calculated using Mathematica and can actually be found faster by replacing the numerator with just 1 since the lowest degree of $t$ after 0 is going to be exactly 201 making it all extra junk.
\end{solution*}



\end{document}