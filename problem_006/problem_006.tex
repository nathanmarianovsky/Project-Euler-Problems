\documentclass[12pt, letterpaper, onecolumn, conference, final]{IEEEtran}

\usepackage[margin = .5in]{geometry}
\usepackage{amsmath}
\usepackage{amsthm}
\usepackage{amssymb}
\usepackage{wasysym}
\usepackage{graphicx}
\usepackage{mathtools}

\title{Project Euler: Problem 6}
\author{Nathan Marianovsky}

\newcommand{\Mod}[1]{\ (\text{mod}\ #1)}

\theoremstyle{definition}
\newtheorem*{problem*}{Problem}
\newtheorem*{solution*}{Solution}

\theoremstyle{plain}

\renewcommand\thesection{\arabic{section}}

\begin{document}

\maketitle

\begin{center}
\fbox{
\begin{minipage}{7.3 in}
\begin{problem*}[Sum Square Difference] 
The sum of the squares of the first ten natural numbers is,
\begin{equation*}
1^2 + 2^2 + \dots + 10^2 = 385
\end{equation*}
The square of the sum of the first ten natural numbers is,
\begin{equation*}
(1 + 2 + \dots + 10)^2 = 3025
\end{equation*}
Hence the difference between the sum of the squares of the first ten natural numbers and the square of the sum is 3025 - 385 = 2640. Find the difference between the sum of the squares of the first one hundred natural numbers and the square of the sum.
\end{problem*}
\end{minipage}}
\end{center}

\vspace{.3cm}
\begin{solution*}
To approach this problem, the following two formulas are needed to make life easier:
\begin{equation*}
\begin{split}
\sum_{i=1}^n i &= \frac{n(n+1)}{2} \\
\sum_{i=1}^n i^2 &= \frac{n(n+1)(2n+1)}{6}
\end{split}
\end{equation*}
where the first equation represents the sum of the first $n$ natural numbers and the second the sum of the squares of the first $n$ natural numbers. This allows for a generalization of $n$ terms, even though this problem specifically wants $n=100$. Using this the difference between the square of the sum and the sum of the squares of the first $n$ natural numbers is:
\begin{equation*}
D = \Big( \sum_{i=1}^n i \Big)^2 - \sum_{i=1}^n i^2 = \Big( \frac{n(n+1)}{2} \Big)^2 - \frac{n(n+1)(2n+1)}{6} = \frac{n^4}{4} + \frac{n^3}{6} - \frac{n^2}{4} - \frac{n}{6}
\end{equation*}
\end{solution*}



\end{document}