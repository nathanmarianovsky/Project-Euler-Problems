\documentclass[12pt, letterpaper, onecolumn, conference, final]{IEEEtran}

\usepackage[margin = .5in]{geometry}
\usepackage{amsmath}
\usepackage{amsthm}
\usepackage{amssymb}
\usepackage{wasysym}
\usepackage{graphicx}
\usepackage{mathtools}

\title{Project Euler: Problem 5}
\author{Nathan Marianovsky}

\newcommand{\Mod}[1]{\ (\text{mod}\ #1)}

\theoremstyle{definition}
\newtheorem*{problem*}{Problem}
\newtheorem*{solution*}{Solution}

\theoremstyle{plain}

\renewcommand\thesection{\arabic{section}}

\begin{document}

\maketitle

\begin{center}
\fbox{
\begin{minipage}{7.3 in}
\begin{problem*}[Smallest Multiple] 
2520 is the smallest number that can be divided by each of the numbers from 1 to 10 without any remainder. What is the smallest positive number that is evenly divisible by all of the numbers from 1 to 20?
\end{problem*}
\end{minipage}}
\end{center}

\vspace{.3cm}
\begin{solution*}
This problem deals with finding the least common multiple of a list of $N$ positive integers. So lets assume that there is a list of these divisors:
\begin{equation*}
d = \{d_1,d_2,\dots,d_N\} \hspace{.3cm} \text{where} \hspace{.3cm} n \in \{2,3,4,\dots\}
\end{equation*}
For this question the list consists of the first 20 positive integers. Now to compute the LCM$(m,n)$, least common multiple, use:
\begin{equation*}
\text{LCM}(m,n) = \frac{mn}{\text{GCD}(m,n)}
\end{equation*}
where GCD$(m,n)$ is the greatest common divisor. This can be generalized to a recursive relation:
\begin{equation*}
\text{LCM}(d_N,d_{N-1},d_{N-2}) = \text{LCM}(d_{N-1},d_{N-2}) \frac{d_N}{\text{GCD}(\text{LCM}(d_{N-1},d_{N-2}),d_N)}
\end{equation*}
Iterating through all of the divisors in question using the above recursive relationship will obtain LCD$(d_1,d_2,\dots,d_N)$.
\end{solution*}



\end{document}