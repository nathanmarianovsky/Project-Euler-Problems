\documentclass[12pt, letterpaper, onecolumn, conference, final]{IEEEtran}

\usepackage[margin = .5in]{geometry}
\usepackage{amsmath}
\usepackage{amsthm}
\usepackage{amssymb}
\usepackage{wasysym}
\usepackage{graphicx}
\usepackage{mathtools}

\title{Project Euler: Problem 45}
\author{Nathan Marianovsky}

\newcommand{\Mod}[1]{\ (\text{mod}\ #1)}

\theoremstyle{definition}
\newtheorem*{problem*}{Problem}
\newtheorem*{solution*}{Solution}

\theoremstyle{plain}

\renewcommand\thesection{\arabic{section}}

\begin{document}

\maketitle

\begin{center}
\fbox{
\begin{minipage}{7.3 in}
\begin{problem*}[Triangular, Pentagonal, and Hexagonal] 
Triangle, pentagonal, and hexagonal numbers are generated by the following formulae:
\begin{center}
\def\arraystretch{1.5}
\begin{tabular}{l l l}
Triangle & $T_n = \frac{n(n+1)}{2}$ & $1, 3, 6, 10, 15, \dots$ \\
Pentagonal & $P_n = \frac{n(3n-1)}{2}$ & $1, 5, 12, 22, 35, \dots$ \\
Hexagonal & $H_n = n(2n-1)$ & $1, 6, 15, 28, 45, \dots$
\end{tabular}
\end{center}
It can be verified that $T_{285} = P_{165} = H_{143} = 40755$. Find the next triangle number that is also pentagonal and hexagonal.
\end{problem*}
\end{minipage}}
\end{center}

\vspace{.3cm}
\begin{solution*}
To approach this, perhaps it might be a good idea to have a way of telling whether a given number is triangular, hexagonal, or pentagonal. To do this, rewrite each of the generating formula into quadratic form and solve for $n$:
\begin{center}
\def\arraystretch{2}
\begin{tabular}{| l | c | c |}
\hline
Sequence & Quadratic Form & Solution \\ \hline
Triangle & $0 = n^2 + n - 2T_n$ & $n = \frac{-1 \pm \sqrt{1 + 8T_n}}{2}$ \\ \hline
Pentagonal & $0 = 3n^2 - n - 2P_n$ & $n = \frac{1 \pm \sqrt{1 + 24P_n}}{6}$ \\ \hline
Hexagonal & $0 = 2n^2 - n - H_n$ & $n = \frac{1 \pm \sqrt{1 + 8H_n}}{4}$ \\ \hline
\end{tabular}
\end{center}
Since each of the sequences consists of strictly positive integers, we can drop the negative radical for all three solutions. Now this problem can be generalized to finding the $n^\text{th}$ number that satisfies all three sequences. If a number does belong to all three sequences then the following conditions will be met:
\begin{center}
\def\arraystretch{2}
\begin{tabular}{| l | c |}
\hline
Sequence & Condition \\ \hline
Triangle & $-1 + \sqrt{1 + 8T_n} \equiv 0 \Mod{2}$ \\ \hline
Pentagonal & $1 + \sqrt{1 + 24P_n} \equiv 0 \Mod{6}$ \\ \hline
Hexagonal & $1 + \sqrt{1 + 8H_n} \equiv 0 \Mod{4}$ \\ \hline
\end{tabular}
\end{center}
So keep iterating through the natural numbers until the $n^\text{th}$ number that satisfies the above congruence relations is found.
\end{solution*}



\end{document}